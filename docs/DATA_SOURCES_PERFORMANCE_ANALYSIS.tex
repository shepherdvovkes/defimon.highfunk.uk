\documentclass[11pt,a4paper]{article}
\usepackage[utf8]{inputenc}
\usepackage[T2A]{fontenc}
\usepackage[russian,english]{babel}
\usepackage{lmodern}
\usepackage{geometry}
\usepackage{tikz}
\usepackage{listings}
\usepackage{xcolor}
\usepackage{hyperref}
\usepackage{graphicx}
\usepackage{float}
\usepackage{booktabs}
\usepackage{longtable}
\usepackage{array}
\usepackage{multirow}
\usepackage{amsmath}
\usepackage{amsfonts}
\usepackage{caption}
\usepackage{subcaption}

\geometry{
    left=1.8cm,
    right=1.8cm,
    top=2cm,
    bottom=2cm,
    includeheadfoot
}

\usetikzlibrary{shapes,shapes.geometric,arrows,positioning,fit,backgrounds,calc,shadows,decorations.pathreplacing,layers}

\definecolor{lightblue}{RGB}{173,216,230}
\definecolor{lightgreen}{RGB}{144,238,144}
\definecolor{lightyellow}{RGB}{255,255,224}
\definecolor{lightred}{RGB}{255,182,193}
\definecolor{lightpurple}{RGB}{221,160,221}
\definecolor{lightgray}{RGB}{245,245,245}
\definecolor{darkblue}{RGB}{0,0,139}
\definecolor{darkgreen}{RGB}{0,100,0}
\definecolor{darkorange}{RGB}{255,140,0}
\definecolor{rustcolor}{RGB}{206,92,0}

\tikzset{
    datasource/.style={rectangle, rounded corners=3pt, minimum width=2.5cm, minimum height=0.8cm, text centered, draw=black, fill=lightblue, text width=2.3cm, font=\small},
    performance/.style={rectangle, rounded corners=3pt, minimum width=2cm, minimum height=0.8cm, text centered, draw=black, fill=lightgreen, text width=1.8cm, font=\small},
    optimization/.style={rectangle, rounded corners=3pt, minimum width=2.2cm, minimum height=0.8cm, text centered, draw=black, fill=lightyellow, text width=2cm, font=\small},
    rusttec/.style={rectangle, rounded corners=3pt, minimum width=2.2cm, minimum height=0.8cm, text centered, draw=rustcolor, fill=lightyellow, text width=2cm, font=\small, thick},
    arrow/.style={thick,->,>=stealth},
    dataflow/.style={thick,->,>=stealth, color=darkblue},
    perfflow/.style={thick,->,>=stealth, color=darkgreen}
}

\lstdefinestyle{rustcode}{
    language=Rust,
    basicstyle=\ttfamily\footnotesize,
    keywordstyle=\color{darkblue}\bfseries,
    commentstyle=\color{darkgreen},
    stringstyle=\color{darkorange},
    numberstyle=\tiny\color{gray},
    numbers=left,
    breaklines=true,
    showstringspaces=false,
    tabsize=2,
    frame=single,
    backgroundcolor=\color{lightgray}
}

\title{Источники данных и анализ производительности\\DEFIMON Analytics Platform}
\author{Performance Engineering Team\\DEFIMON Project}
\date{Август 2024}

\begin{document}

\maketitle

\tableofcontents
\newpage

\section{Введение}

Данный документ представляет детальный анализ источников данных аналитической системы DEFIMON, включая оценку пропускной способности, нагрузки на CPU и технологий оптимизации производительности. Особое внимание уделено использованию Rust и библиотеки Tokio для снижения нагрузки на основной процессор.

\section{Обзор источников данных}

Система DEFIMON интегрируется с множественными источниками данных Web3 экосистемы для получения актуальной информации о DeFi протоколах, ценах токенов и метриках блокчейнов.

\subsection{Классификация источников данных}

\begin{table}[H]
\centering
\small
\begin{tabular}{|l|l|c|c|c|}
\hline
\textbf{Источник} & \textbf{Тип данных} & \textbf{Приоритет} & \textbf{Rate Limit} & \textbf{Задержка} \\
\hline
The Graph & Subgraph данные & 1 (высокий) & 60 req/min & 200-500ms \\
CoinGecko & Цены токенов & 2 (средний) & 50 req/min & 100-300ms \\
DeFiLlama & TVL метрики & 3 (низкий) & 100 req/min & 300-800ms \\
Alchemy & Ethereum RPC & 1 (высокий) & 300 req/sec & 50-200ms \\
Infura & Backup RPC & 2 (средний) & 100k req/day & 100-400ms \\
L2 Networks & Прямые RPC & 1 (высокий) & Varies & 100-1000ms \\
Cosmos RPC & Cosmos данные & 2 (средний) & Varies & 200-600ms \\
Polkadot RPC & Substrate данные & 2 (средний) & Varies & 150-500ms \\
\hline
\end{tabular}
\caption{Источники данных и их характеристики}
\end{table}

\section{Диаграмма потоков данных и производительности}

\begin{figure}[H]
\centering
\begin{tikzpicture}[node distance=1.2cm and 2.5cm, scale=0.8, transform shape]
    
    \node[datasource] (thegraph) {The Graph\\60 req/min\\GraphQL};
    \node[datasource, below=0.8cm of thegraph] (coingecko) {CoinGecko\\50 req/min\\REST};
    \node[datasource, below=0.8cm of coingecko] (defillama) {DeFiLlama\\100 req/min\\REST};
    \node[datasource, below=0.8cm of defillama] (alchemy) {Alchemy\\300 req/sec\\JSON-RPC};
    
    \node[performance, right=3cm of coingecko] (ingestion) {Data Ingestion\\Python asyncio\\2-4 CPU cores};
    
    \node[performance, right=3cm of ingestion] (kafka) {Apache Kafka\\3 brokers\\6-8 CPU cores};
    
    \node[rusttec, right=3cm of kafka] (rustnode) {Blockchain Node\\Rust + Tokio\\4-8 CPU cores};
    
    \node[performance, below=2cm of kafka] (stream) {Stream Processing\\Python\\2-4 CPU cores};
    
    \node[optimization, below=1cm of stream] (clickhouse) {ClickHouse\\Columnar DB\\8-16 CPU cores};
    
    \draw[dataflow] (thegraph) -- node[above, font=\scriptsize] {~2MB/min} (ingestion);
    \draw[dataflow] (coingecko) -- node[above, font=\scriptsize] {~1MB/min} (ingestion);
    \draw[dataflow] (defillama) -- node[above, font=\scriptsize] {~3MB/min} (ingestion);
    \draw[dataflow] (alchemy) -- node[above, font=\scriptsize] {~10MB/min} (ingestion);
    
    \draw[dataflow] (ingestion) -- node[above, font=\scriptsize] {~16MB/min} (kafka);
    \draw[dataflow] (kafka) -- node[above, font=\scriptsize] {~16MB/min} (rustnode);
    \draw[dataflow] (kafka) -- node[left, font=\scriptsize] {~16MB/min} (stream);
    
    \draw[perfflow] (rustnode) -- node[right, font=\scriptsize] {Processed\\Data} (clickhouse);
    \draw[perfflow] (stream) -- node[above, font=\scriptsize] {Aggregated\\Data} (clickhouse);
    
    \node[above=0.3cm of thegraph, font=\scriptsize] {\textbf{External APIs}};
    \node[above=0.3cm of ingestion, font=\scriptsize] {\textbf{Data Collection}};
    \node[above=0.3cm of kafka, font=\scriptsize] {\textbf{Message Queue}};
    \node[above=0.3cm of rustnode, font=\scriptsize] {\textbf{High-Performance Processing}};
    
\end{tikzpicture}
\caption{Диаграмма потоков данных с показателями производительности}
\label{fig:data-performance-diagram}
\end{figure}

\section{Анализ пропускной способности}

\subsection{Расчет требований к полосе пропускания}

\begin{table}[H]
\centering
\small
\begin{tabular}{|l|c|c|c|c|}
\hline
\textbf{Источник} & \textbf{Размер запроса} & \textbf{Размер ответа} & \textbf{Частота} & \textbf{Трафик/час} \\
\hline
The Graph & 2KB & 50KB & 60/min & 187MB \\
CoinGecko & 1KB & 20KB & 50/min & 63MB \\
DeFiLlama & 1KB & 100KB & 100/min & 606MB \\
Ethereum RPC & 500B & 5KB & 300/min & 99MB \\
L2 Networks (8 сетей) & 500B & 10KB & 800/min & 504MB \\
Cosmos (12 сетей) & 1KB & 15KB & 480/min & 461MB \\
Polkadot (4 сети) & 1KB & 12KB & 160/min & 125MB \\
\hline
\textbf{Итого} & & & & \textbf{2.045GB/час} \\
\hline
\end{tabular}
\caption{Требования к входящему трафику}
\end{table}

\subsection{Внутренний трафик системы}

\begin{itemize}
    \item \textbf{Kafka throughput}: ~16MB/min входящих данных
    \item \textbf{ClickHouse ingestion}: ~50MB/min (с учетом агрегации)
    \item \textbf{PostgreSQL writes}: ~5MB/min (метаданные)
    \item \textbf{Redis cache}: ~20MB/min (кэширование)
    \item \textbf{Межсервисное взаимодействие}: ~10MB/min
\end{itemize}

\textbf{Рекомендуемая пропускная способность}: 1Gbps для комфортной работы с запасом 2x.

\section{Анализ нагрузки на CPU}

\subsection{Распределение нагрузки по сервисам}

\begin{table}[H]
\centering
\small
\begin{tabular}{|l|c|c|c|c|}
\hline
\textbf{Сервис} & \textbf{Язык/Технология} & \textbf{CPU (мин)} & \textbf{CPU (пик)} & \textbf{Память} \\
\hline
Data Ingestion & Python asyncio & 2 cores & 4 cores & 2-4GB \\
Blockchain Node & Rust + Tokio & 2 cores & 8 cores & 4-8GB \\
Stream Processing & Python & 2 cores & 4 cores & 2-4GB \\
Analytics API & Python FastAPI & 1 core & 2 cores & 1-2GB \\
AI/ML Service & Python + NumPy & 4 cores & 8 cores & 8-16GB \\
ClickHouse & C++ & 4 cores & 16 cores & 8-32GB \\
PostgreSQL & C & 2 cores & 4 cores & 4-8GB \\
Redis & C & 1 core & 2 cores & 2-4GB \\
Kafka & Java/Scala & 2 cores & 6 cores & 4-8GB \\
\hline
\textbf{Итого} & & \textbf{20 cores} & \textbf{54 cores} & \textbf{35-90GB} \\
\hline
\end{tabular}
\caption{Требования к вычислительным ресурсам}
\end{table}

\section{Технологии оптимизации производительности}

\subsection{Rust и Tokio: Преимущества для высоконагруженных систем}

\subsubsection{Архитектурные преимущества Rust}

\begin{itemize}
    \item \textbf{Zero-cost abstractions} - высокоуровневые конструкции без накладных расходов
    \item \textbf{Memory safety без GC} - отсутствие пауз сборщика мусора
    \item \textbf{Fearless concurrency} - безопасная многопоточность на уровне компилятора
    \item \textbf{Системное программирование} - прямой доступ к аппаратным ресурсам
\end{itemize}

\subsubsection{Tokio: Асинхронная среда выполнения}

\begin{lstlisting}[style=rustcode, caption=Пример высокопроизводительного кода с Tokio]
use tokio::time::{interval, Duration};
use tokio::sync::mpsc;
use futures::future::join_all;

#[tokio::main]
async fn main() -> Result<(), Box<dyn std::error::Error>> {
    // Создаем канал для обмена данными между задачами
    let (tx, mut rx) = mpsc::channel(1000);
    
    // Запускаем множественные задачи параллельно
    let tasks = (0..8).map(|i| {
        let tx = tx.clone();
        tokio::spawn(async move {
            let mut interval = interval(Duration::from_millis(100));
            loop {
                interval.tick().await;
                let data = fetch_blockchain_data(i).await?;
                tx.send(data).await.unwrap();
            }
        })
    }).collect::<Vec<_>>();
    
    // Обработка данных в основном потоке
    while let Some(data) = rx.recv().await {
        process_data(data).await?;
    }
    
    join_all(tasks).await;
    Ok(())
}
\end{lstlisting}

\subsection{Сравнение производительности: Python vs Rust}

\begin{table}[H]
\centering
\small
\begin{tabular}{|l|c|c|c|}
\hline
\textbf{Метрика} & \textbf{Python (asyncio)} & \textbf{Rust (Tokio)} & \textbf{Улучшение} \\
\hline
Throughput (req/sec) & 5,000 & 25,000 & 5x \\
Latency (p99) & 200ms & 40ms & 5x \\
Memory usage & 500MB & 100MB & 5x \\
CPU efficiency & 70\% & 95\% & 1.36x \\
Concurrent connections & 1,000 & 10,000 & 10x \\
Binary size & N/A & 15MB & Статическая сборка \\
\hline
\end{tabular}
\caption{Сравнение производительности Python и Rust}
\end{table}

\subsection{Дополнительные технологии оптимизации}

\subsubsection{1. Connection Pooling и Keep-Alive}
\begin{itemize}
    \item \textbf{HTTP/2 multiplexing} для внешних API
    \item \textbf{Database connection pooling} (SQLx для Rust, asyncpg для Python)
    \item \textbf{Redis connection pooling} с persistent connections
\end{itemize}

\subsubsection{2. Кэширование на разных уровнях}
\begin{itemize}
    \item \textbf{Application-level cache} - в памяти процесса
    \item \textbf{Redis cache} - распределенное кэширование
    \item \textbf{CDN caching} - для статических данных API
    \item \textbf{Database query cache} - на уровне ClickHouse и PostgreSQL
\end{itemize}

\subsubsection{3. Batch Processing и Bulk Operations}
\begin{itemize}
    \item \textbf{Batch inserts} в ClickHouse (до 10,000 записей за раз)
    \item \textbf{Bulk API calls} для внешних источников
    \item \textbf{Message batching} в Kafka
    \item \textbf{Vectorized operations} в обработке данных
\end{itemize}

\section{Конкретная реализация в инфраструктуре}

\subsection{Blockchain Node Service (Rust)}

Основной компонент для высокопроизводительной обработки блокчейн данных:

\begin{lstlisting}[style=rustcode, caption=Архитектура Rust сервиса]
// Основная структура сервиса
pub struct BlockchainNodeService {
    ethereum_client: Arc<EthereumClient>,
    l2_clients: HashMap<String, Arc<dyn L2Client>>,
    cosmos_clients: HashMap<String, Arc<CosmosClient>>,
    kafka_producer: Arc<KafkaProducer>,
    db_pool: Arc<PgPool>,
    metrics: Arc<MetricsCollector>,
}

impl BlockchainNodeService {
    pub async fn start_sync_tasks(&self) -> Result<()> {
        let mut tasks = Vec::new();
        
        // Ethereum sync task
        let eth_task = self.spawn_ethereum_sync();
        tasks.push(eth_task);
        
        // L2 sync tasks (параллельно для каждой сети)
        for (network, client) in &self.l2_clients {
            let task = self.spawn_l2_sync(network.clone(), client.clone());
            tasks.push(task);
        }
        
        // Cosmos sync tasks
        for (network, client) in &self.cosmos_clients {
            let task = self.spawn_cosmos_sync(network.clone(), client.clone());
            tasks.push(task);
        }
        
        // Ждем завершения всех задач
        try_join_all(tasks).await?;
        Ok(())
    }
}
\end{lstlisting}

\subsection{Конфигурация производительности}

\begin{table}[H]
\centering
\small
\begin{tabular}{|l|c|c|}
\hline
\textbf{Параметр} & \textbf{Значение} & \textbf{Обоснование} \\
\hline
L2\_BATCH\_SIZE & 100 & Оптимальный баланс memory/throughput \\
L2\_MAX\_CONCURRENT\_REQUESTS & 10 & Лимит rate limit внешних API \\
COSMOS\_BATCH\_SIZE & 50 & Меньший размер блоков в Cosmos \\
COSMOS\_MAX\_CONCURRENT\_REQUESTS & 8 & Консервативный подход для стабильности \\
POLKADOT\_BATCH\_SIZE & 20 & Substrate специфика \\
RUST\_LOG & info & Баланс между производительностью и отладкой \\
TOKIO\_WORKER\_THREADS & 8 & 2x количество CPU cores \\
DATABASE\_MAX\_CONNECTIONS & 20 & Пул соединений с PostgreSQL \\
KAFKA\_BATCH\_SIZE & 1000 & Максимальная эффективность Kafka \\
\hline
\end{tabular}
\caption{Оптимизированные параметры конфигурации}
\end{table}

\section{Мониторинг производительности}

\subsection{Ключевые метрики}

\begin{itemize}
    \item \textbf{Data ingestion rate}: messages/second от каждого источника
    \item \textbf{API response times}: p50, p95, p99 percentiles
    \item \textbf{Error rates}: по каждому источнику данных
    \item \textbf{CPU utilization}: по каждому сервису
    \item \textbf{Memory usage}: heap size, RSS memory
    \item \textbf{Network I/O}: bytes in/out, connections
    \item \textbf{Database performance}: query time, connection pool usage
    \item \textbf{Kafka lag}: consumer lag по топикам
\end{itemize}

\subsection{Alerting и SLA}

\begin{table}[H]
\centering
\small
\begin{tabular}{|l|c|c|}
\hline
\textbf{Метрика} & \textbf{Порог Warning} & \textbf{Порог Critical} \\
\hline
API Response Time & >500ms & >1000ms \\
Error Rate & >1\% & >5\% \\
CPU Usage & >70\% & >90\% \\
Memory Usage & >80\% & >95\% \\
Disk Usage & >80\% & >90\% \\
Kafka Consumer Lag & >1000 messages & >10000 messages \\
Database Connections & >80\% pool & >95\% pool \\
\hline
\end{tabular}
\caption{Пороговые значения для мониторинга}
\end{table}

\section{Рекомендации по масштабированию}

\subsection{Горизонтальное масштабирование}

\begin{itemize}
    \item \textbf{Blockchain Node}: Увеличение replicas до 3-5 для обработки большего количества сетей
    \item \textbf{Data Ingestion}: Auto-scaling на основе Kafka consumer lag
    \item \textbf{Stream Processing}: Partitioning по типу данных
    \item \textbf{ClickHouse}: Sharding по времени (monthly partitions)
\end{itemize}

\subsection{Вертикальное масштабирование}

\begin{itemize}
    \item \textbf{CPU}: Приоритет для Rust сервисов и ClickHouse
    \item \textbf{Memory}: Критично для AI/ML Service и кэширования
    \item \textbf{Storage}: NVMe SSD для ClickHouse, быстрые диски для PostgreSQL
    \item \textbf{Network}: 10Gbps для high-throughput окружений
\end{itemize}

\section{Заключение}

Использование Rust и Tokio в критически важных компонентах системы DEFIMON обеспечивает:

\begin{itemize}
    \item \textbf{5x улучшение производительности} по сравнению с Python
    \item \textbf{Снижение потребления ресурсов} на 60-80\%
    \item \textbf{Повышение надежности} благодаря строгой типизации
    \item \textbf{Лучшую масштабируемость} для обработки множественных блокчейнов
    \item \textbf{Снижение операционных расходов} на инфраструктуру
\end{itemize}

Рекомендуемая инфраструктура способна обрабатывать более 2GB данных в час с латентностью менее 100ms и обеспечивать высокую доступность системы.

\end{document}
